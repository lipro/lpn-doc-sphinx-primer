\begin{tikztimingtable}[timing/wscale=2]

%% %%%%%%%%%%%%%%%%%%%%%%%%%%%%%%%%%%%%%%%%%%%%%%%%%%%%%%%%%%%%%%%
%%
%% My Customized Styles
%%

\def\strut{\large\vphantom{/}}

%% %%%%%%%%%%%%%%%%%%%%%%%%%%%%%%%%%%%%%%%%%%%%%%%%%%%%%%%%%%%%%%%
%%
%% Drawing
%%

%      section A
\\  % empty line for time scale
RefClock\_150MHz & h19{c}                                  \\
%      section B
                 & \divider{Transceiver Interface}         \\
PowerDown        & [very thick,violet]lN(BA1)h5HhN(BA2)l3L \\
PowerDownDone    & [very thick,orange]3LN(BB1)3H3HN(BB2)1L \\
%      section C
                 & \divider{Synchronized Interface}        \\
CC\_PowerDown    & [thick,orange]2LN(CA1)6HN(CA2)2L        \\
%
% there must NOT be an uncommented line before \extracode!
%
\extracode
    \makeatletter
    \tableheader{Signal Name}{Waveform}\tablerules
    \tablerules
    % time scale
    \pgfmathsetmacro\twidthnew{\twidth/2}
    \pgfmathsetmacro\nrowsnew{76}
    \draw[->,thick] (0,0) -- +(\twidth+1,0);
    \foreach \n in {0,1,...,\twidthnew}
        \pgfmathsetmacro\nnew{int(\n*2)}
        \draw(\nnew,-0.2) -- +(0,.4) node[above,inner sep=2pt] {\scalebox{0.75}{\n}};
    % interconnection
    \draw[blue,->](BA1)to[out=0,in=180](CA1);
    \draw[blue,->](CA1)to[out=0,in=180](BB1);
    \draw[blue,->](BA2)to[out=0,in=180](CA2);
    \draw[blue,->](CA2)to[out=0,in=180](BB2);
    % table grid
    \begin{background}[lightgray!50,semitransparent,semithick]
        \horlines{1,...,8}
        \vertlines{0,2,...,6,8,10,...,\twidth}
    \end{background}
    % draw dividers
    \dividers
    
%%
%%
%% %%%%%%%%%%%%%%%%%%%%%%%%%%%%%%%%%%%%%%%%%%%%%%%%%%%%%%%%%%%%%%%

\end{tikztimingtable}

%Local variables:
% coding: utf-8
% mode: text
% mode: rst
% End:
% vim: fileencoding=utf-8 filetype=tex :
